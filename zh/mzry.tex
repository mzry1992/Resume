\documentclass[11pt,a4paper,sans]{moderncv}

\usepackage{xunicode}
\usepackage{amsmath}

% moderncv themes
\moderncvstyle{classic}
\moderncvcolor{black}

%%%%%%%%%%%%%%%%%%%%%%%%%%%%%%%%%%
\renewcommand*{\namefont}{\fontsize{50}{52}\mdseries\upshape}
%%%%%%%%%%%%%%%%%%%%%%%%%%%%%%%%%%

%\usepackage[scale=0.75]{geometry}
\usepackage{geometry}

\usepackage{xeCJK}
%\setsansfont{Monaco}

\setCJKmainfont{楷体}
\setCJKsansfont{黑体}
\setCJKmonofont{楷体}

% personal data
\firstname{李}
\familyname{昀}
%\title{Resumé title (optional)} 
\address{四川省成都市高新区(西区)西源大道2006号}{电子科技大学清水河校区}{邮编:611731}
\mobile{+86~136~7903~3612}                
%\phone{+2~(345)~678~901}                
%\fax{+3~(456)~789~012}                 
\email{muziriyun@gmail.com}                  
\homepage{www.mzry1992.com}
%\extrainfo{additional information}   
%\photo[64pt][0.4pt]{picture}        
%\quote{Some quote (optional)}      

\begin{document}
\makecvtitle
%\maketitle

\section{教育经历}
\cventry{2010--至今}{\textsf{学士}}{电子科技大学}{通信工程}
{\textit{加权平均分3.07}}
%{}
{\textsf{主干课程:}
	离散数学、电路分析基础、模拟电路基础、信号与系统、软件技术基础、数字逻辑设计及应用、电磁场与波、信息论导论、随机信号分析、计算机通信网、
	微型计算机系统原理及接口应用、数字信号处理、通信原理、移动通信系统。}
\cventry{2007-2010}{\textsf{高中}}{江西省南康中学}{}
{}
{获全国青少年信息学奥林匹克联赛省级一等奖保送至电子科技大学。同时还获得哈尔滨工业大学、华南理工大学、大连理工大学的保送生资格。}

%\section{Master thesis}
%\cvitem{title}{\emph{Title}}
%\cvitem{supervisors}{Supervisors}
%\cvitem{description}{Short thesis abstract}

\section{相关经验}
\subsection{项目经验}
\cventry{2012--至今}{\textsf{数据库 JavaEE}}{UESTC ACM-ICPC Online Judge前端相关开发}{}{}{
	\begin{description}
		\item[\textsf{项目主页}] \texttt{https://gitcafe.com/UESTC\_ACM/cdoj}
		\item[\textsf{项目内容}] ~\\
			这个项目实现了一个Online judge用来提供在线测评服务并被用来帮助校ACM/ICPC队伍组织和管理各种训练计划和相关比赛。
			此系统还将运用于大学生计算机课程上机实验。
		\item[\textsf{开发平台}] Jdk1.7, J2ee platform, maven
		\item[\textsf{开发工具}] IntelliJ IDEA, MySQL, chrome
		\item[\textsf{相关框架}] Struts2, Spring, Hibernate, Sitemesh, Bootstrap, jQuery
		\item[\textsf{个人职责}] 负责整体前端UI设计与实现,以及与后台的结合,还负责对系统进行整体测试。
	\end{description}
}
\subsection{竞赛经验}
\cventry{2010--至今}{电子科技大学ACM/ICPC集训队}{}{}{}{
	\begin{itemize}
		\item 深入学习研究了各种算法如图论、动态规划,常用数据结构如动态树、KD树、Splay,拥有较强的代码能力。
		\item 在TopCoder公司举办的SRM中排名前$8\%$(2013-5-5),该比赛要求快速地写出正确高效的代码和找出其他选手代码中bug。
		\item 在ACM/ICPC的竞赛中,得到了大量团队合作解题的经验。
	\end{itemize}
}
\cventry{2013}{\textsf{NOI2013全国青少年信息学奥林匹克竞赛四川代表队选拔赛}}{命题工作}{}{}{}
\cventry{2012}{\textsf{第四届四川省大学生程序设计竞赛}}{命题与裁判工作}{}{}{}
\cventry{2012}{\textsf{NOI2012全国青少年信息学奥林匹克竞赛四川代表队选拔赛}}{命题与裁判工作}{}{}{}

\section{荣誉}
\subsection{ACM-ICPC}
\cvline{2012}{第37届ACM-ICPC亚洲区预选赛成都赛区 \textsf{金牌}}
\cvline{2012}{第37届ACM-ICPC亚洲区预选赛金华赛区 \textsf{金牌(亚军)}}
\cvline{2012}{四川大学程序设计竞赛 \textsf{冠军}}
\cvline{2012}{第10届电子科技大学程序设计竞赛 \textsf{亚军}}
\cvline{2011}{第36届ACM-ICPC亚洲区预选赛成都赛区 \textsf{金牌(亚军)}}
\cvline{2011}{第36届ACM-ICPC亚洲区预选赛北京赛区 \textsf{金牌}}
\cvline{2011}{第三届四川省大学生程序设计竞赛 \textsf{亚军}}
\cvline{2011}{第36届ACM-ICPC亚洲区预选赛福州赛区资格赛 \textsf{金牌}}
\cvline{2010}{第35届ACM-ICPC亚洲区预选赛福州赛区 \textsf{银牌}}
\cvline{2010}{第35届ACM-ICPC亚洲区预选赛杭州赛区 \textsf{银牌}}

\subsection{证书}
\cvline{2012}{全国计算机等级考试四级网络工程师}

\section{语言技能}
\cventry{英语}{全国大学英语四级考试}{通过}{}{}{}

\section{计算机技能}
\cvitem{\textsf{操作系统}}{linux(arch linux), OS X, windows}
\cvitem{\textsf{团队开发}}{Subversion, Git}
\cvitem{\textsf{数据库}}{MySQL}
\cvitem{\textsf{编程语言}}{C, C++, Java(Java EE), Javascript, C\#}
\cvitem{\textsf{开发工具}}{Vim, IntelliJ IDEA, Visual Studio}
\cvitem{\textsf{其它工具}}{\LaTeX}

\section{个人能力}
\cvlistitem{拥有在短暂的时间内快速学习和解决问题。}
\cvlistitem{拥有良好的思维能力和有效的沟通能力。}
\cvlistitem{对新技术有不错的自学能力。}

\end{document}
