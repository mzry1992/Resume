\documentclass[11pt,a4paper,sans]{moderncv}

\usepackage{xunicode}
\usepackage{amsmath}

% moderncv themes
\moderncvstyle{banking}
\definecolor{color0}{rgb}{0,0,0}% black
\definecolor{color1}{rgb}{0.95,0.20,0.20}% red
\definecolor{color2}{rgb}{0.0,0.0,0.0}% dark grey
%\moderncvcolor{red}

%%%%%%%%%%%%%%%%%%%%%%%%%%%%%%%%%%
\renewcommand*{\namefont}{\fontsize{50}{52}\mdseries\upshape}
%%%%%%%%%%%%%%%%%%%%%%%%%%%%%%%%%%

%\usepackage[top=3.2cm, bottom=3.2cm, left=3.2cm, right=3.2cm]{geometry}
\usepackage[scale=0.76]{geometry}

\usepackage{xeCJK}
%\setsansfont{Monaco}

\setCJKmainfont{楷体}
\setCJKsansfont{黑体}
\setCJKmonofont{楷体}

% personal data
\firstname{李}
\familyname{昀}
%\title{Resumé title (optional)} 
\address{四川省成都市高新区(西区)西源大道2006号}{电子科技大学清水河校区}{邮编:611731}
\mobile{+86~136~7903~3612}                
%\phone{+2~(345)~678~901}                
%\fax{+3~(456)~789~012}                 
\email{muziriyun@gmail.com}                  
\homepage{www.mzry1992.com}
%\extrainfo{additional information}   
%\photo[64pt][0.4pt]{picture}        
%\quote{Some quote (optional)}      

\begin{document}
\makecvtitle
%\maketitle

\section{个人信息}
\cvitem{姓名}{李昀}
\cvitem{年龄}{20(1992-10-5)}
\cvitem{毕业时间}{2014-6}
\cvitem{就业意向}{算法工程师}

\section{教育经历}
\cventry{2010--至今}{通信工程专业}{\textsf{电子科技大学}}{四川省成都市}
{工学学士}
{加权平均分3.07}
{}
%{主干课程:
%	离散数学、电路分析基础、模拟电路基础、信号与系统、软件技术基础、数字逻辑设计及应用、电磁场与波、信息论导论、随机信号分析、计算机通信网、
%	微型计算机系统原理及接口应用、数字信号处理、通信原理、移动通信系统。}

~\\
\cventry{2007-2010}{高中}{\textsf{江西省南康中学}}{江西省南康市}
{}
{2009年获第十五届全国青少年信息学奥林匹克联赛一等奖保送至电子科技大学。同时还获得哈尔滨工业大学、华南理工大学、大连理工大学的保送生资格。}

%\section{Master thesis}
%\cvitem{title}{\emph{Title}}
%\cvitem{supervisors}{Supervisors}
%\cvitem{description}{Short thesis abstract}

\section{相关经验}
\subsection{项目经验}
\cventry{2012--至今}{数据库 JavaEE}{\textsf{UESTC ACM-ICPC Online Judge前端相关开发}}{}{}{
	这个项目实现了一个Online judge用来提供在线测评服务并被用来帮助校ACM/ICPC队伍组织和管理各种训练计划和相关比赛。
	此系统还将运用于大学生计算机课程上机实验。
	\begin{description}
		\item[\textsf{项目主页}] \texttt{https://gitcafe.com/UESTC\_ACM/cdoj}
		\item[\textsf{开发平台}] Jdk1.7, J2ee platform, maven
		\item[\textsf{开发工具}] IntelliJ IDEA, MySQL, chrome
		\item[\textsf{相关框架}] Struts2, Spring, Hibernate, Sitemesh, Bootstrap, jQuery
		\item[\textsf{个人职责}] 负责整体前端UI设计与实现,以及与后台的结合,还负责对系统进行整体测试。
	\end{description}
}
\subsection{竞赛经验}
\cventry{2010--至今}{队员}{\textsf{电子科技大学ACM/ICPC集训队}}{}{}{
	\begin{itemize}
		\item 深入学习研究了各种算法如图论、动态规划,常用数据结构如动态树、KD树、Splay,拥有较强的代码能力。
		\item 在TopCoder公司举办的SRM中排名前$8\%$(2013-5-5),该比赛要求快速地写出正确高效的代码和找出其他选手代码中bug。
		\item 在ACM/ICPC的竞赛中,积累了大量团队合作解题的经验,有很强的团队意识。
	\end{itemize}
}
\cventry{2012--2013}{命题与裁判工作}{\textsf{NOI2012/2013 全国青少年信息学奥林匹克竞赛四川代表队选拔赛}}{}{}{
	\begin{itemize}
		\item 国内最高水平信息学竞赛省队队员选拔赛
		\item 要在5小时内解决3道问题,难度极大
		\item 由于四川省内中学间信息学竞赛水平差异较大,命题难度高
	\end{itemize}
}

~\\
\cventry{2012}{命题与裁判工作}{\textsf{第四届四川省大学生程序设计竞赛}}{}{}{
	\begin{itemize}
		\item 由于四川省大学生程序设计竞赛平均水平较低,需要设计出难度不高但是富有价值的问题
	\end{itemize}
}

\section{荣誉}
\subsection{ACM-ICPC}
\cvline{2012}{第37届ACM-ICPC亚洲区预选赛(成都赛区) \textsf{金牌}}
\cvline{2012}{第37届ACM-ICPC亚洲区预选赛(金华赛区) \textsf{金牌(亚军)}}
\cvline{2012}{“恒生电子杯”四川大学第十一届程序设计竞赛 \textsf{冠军}}
\cvline{2012}{第十届电子科技大学程序设计竞赛暨西南地区高校邀请赛 \textsf{亚军}}
\cvline{2011}{第36届ACM-ICPC亚洲区预选赛(成都赛区) \textsf{金牌(亚军)}}
\cvline{2011}{第36届ACM-ICPC亚洲区预选赛(北京赛区) \textsf{金牌}}
\cvline{2011}{第三届四川省大学生程序设计竞赛 \textsf{亚军}}
\cvline{2011}{第36届ACM-ICPC亚洲区预选赛(福州赛区)入围晋级赛暨第二届福建省大学生程序设计竞赛 \textsf{金牌}}
\cvline{2010}{第35届ACM-ICPC亚洲区预选赛(福州赛区) \textsf{银牌}}
\cvline{2010}{第35届ACM-ICPC亚洲区预选赛(杭州赛区) \textsf{银牌}}

\section{专业技能}
\subsection{语言技能}
\cvitem{\textsf{全国大学英语四级考试}}{通过}

\subsection{计算机技能}
\cvitem{\textsf{操作系统}}{linux(arch linux), OS X, windows}
\cvitem{\textsf{团队开发}}{Subversion, Git}
\cvitem{\textsf{编程语言}}{C, C++, Java(Java EE), Javascript, C\#}
\cvitem{\textsf{开发工具}}{Vim, IntelliJ IDEA, Visual Studio}
\cvitem{\textsf{其它工具}}{MySQL, \LaTeX}
\cvitem{\textsf{相关证书}}{全国计算机等级考试四级网络工程师}

\subsection{其它技能}
\cvitem{\textsf{问题解决}}{能在较短时间内独立思考、周密分析问题并形成解决问题的思路}
\cvitem{\textsf{学习能力}}{对新技术有浓厚的兴趣,有很强的自学能力}
\cvitem{\textsf{团队合作}}{拥有良好的沟通能力,有很强的团队意识}

\end{document}
